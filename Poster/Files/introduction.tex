\title{Introducción}
\hspace{-0.cm}
\begin{minipage}{0.53\linewidth}
  Monterrey y su área metropolitana conforman la 3$^a$ región más poblada
  de México y una de las de mayor deterioro en su Calidad del Aire en las últimas décadas. Por su ubicación geográfica y condiciones atmosféricas, la radiación solar alcanza niveles altos casi todo el año. Conocer la intensidad solar a nivel del suelo nos permite estimar los componentes que la atenúan y también evaluar sus efectos biológicos. Presentamos un análisis de la irradiancia solar Vis+NIR* medida en el periodo 2015-2018 en las estaciones Noreste (NE) y Noroeste (NO) del Sistema Integral de Monitoreo Ambiental (SIMA) de Nuevo León. Las mediciones bajo un cielo libre de nubes fueron las referencias para aproximación de los modelos TUV 5.3.2 y SMARTS 2.9.5$^{\left[1,2,3\right]}$.\\
  \verde{$\leftarrow$Imagen satélital del área metropolitana de Monterrey y localización de estaciones Noreste y Noroeste del SIMA}
\end{minipage}