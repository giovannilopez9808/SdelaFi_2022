\title{Metodologia}
\begin{minipage}{0.81\linewidth}
    La irradiancia solar Vis+NIR* del SIMA se midió con un piranómetro MetOne096 de sensibilidad entre $\left[400,1100\right]$nm. De ellas se seleccionan días despejados en el periodo 2015-2018. Por otro lado, la Ecuación de Transferencia Radiativa:
    \begin{equation*}
        dL_{\lambda}=\sigma_e \ \left(z\right) \left(\cdots - \frac{\omega\left(z\right)}{4\pi} \left[I_{0\lambda}p\left(\vec{s*},\vec{s};z \right)exp\left(-\int\limits_z^\infty \frac{\sigma_e(z')}{cos\theta^*(z')}dz' \right)\cdots \int\limits_{4\pi}\int\limits_{4\pi} L_{\lambda}(\vec{s'};z)p(\vec{s'},\vec{s};z)d^2\omega' \right]\right)\frac{dz}{cos\theta}
    \end{equation*}
    es resuelta por los modelos SMARTS y TUV$^{\left[2,5\right]}$ para obtener la irradiancia solar espectral (L$_\lambda$). Cada modelo se ejecuta para una fecha y hora del día con los siguientes valores de entrada:

    \changefontsizes{9.5pt}
    \begin{tabular}{|c|c|c|c|c|c|c|c|}
        \hline
        Modelo & [Lat, Lon, a.s.n.m] NE/NO           & Reflectividad & O$_3$ col       & NO$_2$ col DU        & Exp.               & Albedo de disp.   & AOD$_{550nm}$                                     \\
               &                                     & de suelo      &                 &                      & Angström           & simple de aerosol &                                                   \\ \hline
        TUV    & \multirow{2}{*}{25.75,-100.25,512m} & 0.06          & OMI-NASA DU     & \multirow{2}{*}{0.1} & \multirow{2}{*}{1} & 0.87              & \multirow{2}{*}{(variable)$^{\left[4,5 \right]}$} \\ \cline{1-1}
        SMARTS &                                     & Concreto      & OMI-NASA atm-cm &                      &                    & urbano            &                                                   \\ \hline
    \end{tabular}\vspace{0.2cm}\\
    \changefontsizes{12pt}
    Con un código propio se integró L$_\lambda$ entre $\left[400,1100\right]$nm para SMARTS y para TUV entre $\left[400,1000\right]$nm añadiendo una fracción aproximada a la integral en el rango espectral coincidente. Luego, la irradiancia Vis+NIR* se ajustó hasta que el AOD$_{500nm}$ logre una diferencia relativa (DR) $<$ 5\% al mediodía solar entre medición y modelo.
\end{minipage}
\hspace{0.3cm}
\begin{minipage}{0.18\linewidth}
    \centering
    % \includegraphics[scale=0.16]{instru.eps}
    \verde{Instrumento de medición de la estación SIMA}
\end{minipage}